\documentclass[%
	a4paper,%             A4 Papier
	oneside,%             Einseitig
	11pt%                 Grössere Schrift, besser lesbar am bildschrim
]{memoir}

% ===== LaTex =====
\usepackage[utf8]{inputenc}
\usepackage[english]{babel}

% ===== XeLaTex =====
%\usepackage{xltxtra}
%\usepackage{polyglossia}
%\setdefaultlanguage[spelling=new,latesthyphen=true]{german}

% ===========================
% ===== Common settings =====
% ===========================

\setcounter{tocdepth}{2}
\maxtocdepth{subsection}
\setcounter{secnumdepth}{3}
\maxsecnumdepth{subsubsection}

\setlength{\parindent}{0pt}
\setlength{\parskip}{1ex plus 0.5ex minus 0.2ex}
\OnehalfSpacing % Zeilenabstand

\usepackage{color}
\usepackage{todonotes}

% ===== For monotypes =====
%\usepackage{courier}
%\DeclareFontFamily{T1}{lmtt}{} 
%\DeclareFontShape{T1}{lmtt}{m}{n}{<-> ec-lmtl10}{} 
%\DeclareFontShape{T1}{lmtt}{m}{\itdefault}{<-> ec-lmtlo10}{} 
%\DeclareFontShape{T1}{lmtt}{\bfdefault}{n}{<-> ec-lmtk10}{} 
%\DeclareFontShape{T1}{lmtt}{\bfdefault}{\itdefault}{<-> ec-lmtko10}{} 
\DeclareFontShape{OT1}{cmtt}{bx}{n}{<5><6><7><8><9><10><10.95><12><14.4><17.28><20.74><24.88>cmttb10}{}

% ===== For bold small caps =====
\usepackage{bold-extra}


% ===== Captions =====
\usepackage[small]{caption}

% ===== Page Layout =====
\usepackage{geometry}
\geometry{a4paper, bottom=3cm} 

% ===== Formatting =====
%\usepackage{ulem} % Underlining

% ===== Lists / Index / Nomenclature =====
%\usepackage{makeidx}
%\makeindex

%\usepackage[german]{nomencl}
%\makenomenclature
%\renewcommand{\nomlabel}[1]{\textbf{#1}}
%\renewcommand{\nomname}{Glossar}

% ===== Wrapping / Positioning =====
\usepackage{wrapfig}

% ===== Tables =====
%\usepackage{multirow} % spanning for tables

% ===== Citations / Bib =====
\usepackage[numbers]{natbib}
%\bibpunct{}{}{,}{n}{}{;}





% ===== Custom Commands =====
\newcommand{\textprog}[1]{\textit{#1}}

\newcommand{\myLink}[1]{\textbf{#1}}

\newcommand{\myRef}[1]{\textbf{\hyperref[#1]{~\ref*{#1}: ~\nameref{#1}}}}
\newcommand{\myRefChapter}[1]{\textbf{\hyperref[#1]{Chapter ~\ref*{#1}: ~\nameref{#1}}}}
\newcommand{\myRefSection}[1]{\textbf{\hyperref[#1]{Section ~\ref*{#1}: ~\nameref{#1}}}}

\newcommand{\myRefFigure}[1]{\textbf{\hyperref[#1]{Figure ~\ref*{#1}}}}
\newcommand{\myRefFigureName}[1]{\textbf{\hyperref[#1]{Figure ~\ref*{#1}: ~\nameref{#1}}}}

\newcommand{\myRefListing}[1]{\textbf{\hyperref[#1]{Listing ~\ref*{#1}}}}

\newcommand{\myNameref}[1]{\textbf{\nameref{#1}}}

\newcommand{\myNomRef}[2]{\textbf{\textsl{\hyperref[#2]{#1}}}}

\newcommand{\myBookCiteTook}[1]{\textsuperscript{[entnommen aus \citeauthor{#1} \citeyear{#1}]}}

\newcommand{\myBookCite}[2]{\footnote{[\cite{#1}] \citeauthor*{#1} \citeyear{#1}, #2}}
\newcommand{\myBookCiteSense}[2]{\footnote{sinngemäß nach [\cite{#1}] \citeauthor*{#1} \citeyear{#1}, #2}}

\newcommand{\myBookCiteNoP}[1]{\footnote{[\cite{#1}] \citeauthor*{#1} \citeyear{#1}}}
\newcommand{\myBookCiteSenseNoP}[1]{\footnote{sinngemäß nach [\cite{#1}] \citeauthor*{#1} \citeyear{#1}}}

%\newcommand{\myWebCiteNoP}[1]{\footnote{[\cite{#1}] \citetalias{#1}}}
%\newcommand{\myWebCiteSenseNoP}[1]{\footnote{sinngemäß nach [\cite{#1}] \citetalias{#1}}}

%\newcommand{\myWebCiteSense}[2]{\footnote{sinngemäß nach [\cite{#1}] \citetalias{#1}, #2}}
%\newcommand{\myWebCite}[2]{\footnote{[\cite{#1}] \citetalias{#1}, #2}}

\usepackage{array,booktabs,arydshln,xcolor}
\newcommand\VRule[1][\arrayrulewidth]{\vrule width #1}

\newcommand{\myLBHighlight}[1]{\vspace{5mm}\hspace{5mm}\begin{tabular}{c!{\color{cyan}\VRule[2pt]}p{0.9\textwidth}}
 & #1
\end{tabular}\vspace{5mm}}

\newcommand{\mySCName}[1]{\texttt{#1}}
\newcommand{\mySCNameImp}[1]{\textbf{\texttt{#1}}}
\newcommand{\myProperName}[1]{\textsc{#1}}
\newcommand{\myProperNameImp}[1]{\textsc{\textbf{#1}}}
\newcommand{\myAuthorName}[1]{#1}
\newcommand{\myAuthorNameImp}[1]{\textsc{#1}}
\newcommand{\myBookName}[1]{#1}
\newcommand{\myBookNameImp}[1]{\textsc{#1}}
\newcommand{\mySCString}[1]{\textcolor{gray}{''#1''}}
\newcommand{\mySlang}[1]{''#1''}
\newcommand{\myInlineQuote}[1]{''#1''}

%\newcommand{\mySCName}[1]{\textcolor{blue}{\texttt{#1}}}
%\newcommand{\mySCNameImp}[1]{\textcolor{blue}{\textbf{\texttt{#1}}}}

%\newcommand{\myProperName}[1]{\textcolor{red}{\textsc{#1}}}
%\newcommand{\myProperNameImp}[1]{\textcolor{red}{\textsc{\textbf{#1}}}}

%\newcommand{\myAuthorName}[1]{\textcolor{green}{#1}}
%\newcommand{\myAuthorNameImp}[1]{\textcolor{green}{\textsc{#1}}}

%\newcommand{\myBookName}[1]{\textcolor{blue}{#1}}
%\newcommand{\myBookNameImp}[1]{\textcolor{blue}{\textsc{#1}}}

%\newcommand{\mySCString}[1]{\textcolor{orange}{''#1''}}

%\newcommand{\mySlang}[1]{\textcolor{blue}{''#1''}}

%\newcommand{\myInlineQuote}[1]{\textcolor{purple}{''#1''}}


% ===== lstlisting: JavaScript =====
% ===== Source-Code =====
\usepackage{listings} %bindet das Paket Listings ein

\lstset{	
%	numbers=none, 					% keine Zeilennummern
%	tabsize=2, 						% Tabulatorgrösse: 2 Zeichen
%	breaklines=true, 				% zu lange Zeilen werden umbrochen
%	aboveskip=1em, 					% Abstand nach oben
%	belowskip=0.3em, 				% Abstand nach unten
%	basicstyle=\tiny, 				% Schriftgrösse small, Typewriter-Font
%	framerule=1pt, 					% keinen Rand
%	backgroundcolor=\color{mygray}, % helles grau als Hintergrund
%	framexrightmargin=0.3em, 		% Hintergrund ragt leicht in den Seitenrand
%	framexleftmargin=0.3em, 		% Hintergrund ragt leicht in den Seitenrand
%	captionpos=t 					% Beschriftung ist unterhalb
%	columns=fullflexible 			% damit Quellcode einfach rauskopiert werden kann
}

\renewcommand{\lstlistlistingname}{Source code listing}
\renewcommand{\lstlistingname}{Source code}

\lstset{
  literate={ö}{{\"o}}1
           {ä}{{\"a}}1
           {ü}{{\"u}}1
           {ß}{{\ss}}1
}

% defining colors
\definecolor{lightgray}{rgb}{.9,.9,.9}
\definecolor{darkgray}{rgb}{.4,.4,.4}
\definecolor{mygreen}{rgb}{0.2, 0.8, 0.2}
\definecolor{myorange}{rgb}{1.0, 0.5, 0}

% defining JavaScript
\lstdefinelanguage{JavaScript}{
  keywords={typeof, new, true, false, catch, function, return, null, catch, switch, var, if, in, while, do, else, case, break},
  sensitive=false,
  comment=[l]{//},
  morecomment=[s]{/*}{*/},
  morestring=[b]',
  morestring=[b]"
}

% loading languages
\lstloadlanguages{C++, JavaScript}

% settings
\lstset{
   extendedchars=true,
   basicstyle=\footnotesize\ttfamily,
   showstringspaces=false,
   showspaces=false,
   numbers=left,
   numberstyle=\footnotesize,
   %numbersep=9pt,
   tabsize=2,
   breaklines=true,
   showtabs=false,
   captionpos=t,
   %colors
   %backgroundcolor=\color{lightgray},
   commentstyle=\color{mygreen}\ttfamily,
   stringstyle=\color{myorange}\ttfamily,
   keywordstyle=\color{blue}\bfseries,
   identifierstyle=\color{black},
   %margins
   xleftmargin=19pt,
}

%caption
\usepackage{caption}
\DeclareCaptionFont{white}{\color{white}}
\DeclareCaptionFormat{listing}{\colorbox[cmyk]{0.43, 0.35, 0.35,0.01}{\parbox{\textwidth}{\hspace{15pt}#1#2#3}}}
\captionsetup[lstlisting]{format=listing,labelfont=white,textfont=white, singlelinecheck=false, margin=0pt, font={bf,footnotesize}}

%\lstset{
%  literate={ö}{{\"o}}1
%           {ä}{{\"a}}1
%           {ü}{{\"u}}1
%           {ß}{{\ss}}1
%}

% ===== Links =====
\usepackage[%
	colorlinks, 		% verwende farbige Links
	linkcolor=black, 	% Linkfarbe ist blau
	bookmarks, 			% erstelle Bookmarks der Links
	bookmarksopen, 		% Bookmarks werden beim Öffnen des Dokumentes ebenfalls geöffnet
	urlcolor=black, 	% Hyperlinks sind blau 
	bookmarksnumbered, 	% Bookmarks sind nummeriert
	final 				% Endversion 
]{hyperref}

\hypersetup{
    colorlinks,%
    citecolor=black,%
    filecolor=black,%
    linkcolor=black,%
    urlcolor=black
}