\chapter{Language binding between C++ and Spidermonkey}\todo{properName}
\label{chap:LanguageBindingCPPJS}

This chapter shows how \myProperName{C++} elements like declarations and types can be exposed to \myProperName{Mozilla Spidermonkey} to be used from scripts. \myProperName{Spidermonkey} is the name of \myProperName{Mozilla}'s implementation of the \myProperName{ECMAScript} programming language standard, which is most commonly known as \myProperName{JavaScript}. The terms \myProperName{JavaScript} and \myProperName{ECMAScript} will be used interchangebly, whereas \myProperName{Spidermonkey} refers to the concrete \myProperName{Mozilla} implementation.\\
Binding \myProperName{C++} elements to other programming languages like \myProperName{Python} or \myProperName{Lua} or other \myProperName{JavaScript} implementations, e.g. \myProperName{V8}\todo{ref}, will be similar in many aspects - depending on the elements existing in the target language.

The information in this chapter presumes knowledge about the \myProperName{C++} programming language, its elements, class-based inheritance and type system.

\section{\myProperName{JavaScript}/\myProperName{ECMAScript}}

\myProperName{JavaScript} is a weakly typed object-oriented programming language with automatic garbage collection. 

As it is a feature-rich language, only the most important aspects will be presented in this section. More information can be found in \myProperName{Mozilla}'s \myProperName{JavaScript} tutorial\todo{ref} or in various books like \myAuthorName{David Flanagan}'s \myBookName{JavaScript: The Definite Guide} or \myAuthorName{Douglas Crockford}'s \myBookName{JavaScript: The Good Parts}.

There are 6 basic types that a value can be of: boolean, number, string, \mySCName{undefined}, \mySCName{null} and object.\\
Number, string, and boolean are primitive types. So are \mySCName{null} and \mySCName{undefined}, which both represent the absence of value. All are immutable.\\
Objects are basically mutable collections of name-value-pairs, known as properties or \mySlang{members}. They can be seen as hash-maps or associative arrays. Object properties can be accessed using bracket-syntax (e.g. \mySCName{var foo["bar"] = 3;}) or dot-notation (e.g. \mySCName{var foo.bar = 3;}). An \mySCName{Array} is a special type of \myProperName{JavaScript} object.\\
Functions are special objects that have source code associated and thus can be invoked. Being objects, functions can have properties and can even be passed as parameters.

\SingleSpacing
\begin{lstlisting}[language=JavaScript, caption=Types in \myProperName{JavaScript}]
// primitive types
var aNumber = 34.9;
var aBoolean = true;
var aString = "Some text!";
var aNullValue = null;
var anUndefinedValue; // standard type for variables that have 
                      // no value assigned
var anotherUndefinedValue = undefined;

// object types
var anObject = {};
anObject.foo = "bar";
anObject.subObject = { foo: "bar"};

var anArray = [];

function aFunction(param)
{
	return param * 3;
}

aFunction.someProperty = "Functions are objects, too!";
\end{lstlisting}
\OnehalfSpacing

When being passed as parameters or assigned to variables, primitive values are passed by value / copy-assigned, whereas objects (including functions) are passed/assigned by reference.\\
\myProperName{JavaScript} thus has no notion of pointers and changes made to primitive values passed as function parameters will not have any effect on the original value.

There are two kinds of properties: value properties and accessor properties. A value property simply contains any type of value (e.g. a number or function). Accessor properties have a getter and setter function associated that can perform additional computation when getting or setting a value. They are used just like normal data properties.

\SingleSpacing
\begin{lstlisting}[language=JavaScript, caption=Types in \myProperName{JavaScript}, label=JSTypes]
var obj = {
	// value properties
	aValue: 3,
	aValue_Function: function(){ this.aValue *= 2; },
	
	// an accessor property
	get accessorProp(){ return this.aValue + 3; },
	set accessorProp(val){ this.aValue = val - 3; }
};

obj.aValue = 4;        // --> obj.aValue is 4
obj.aValue_Function(); // --> obj.aValue is 8
obj.accessorProp = 6;  // --> obj.aValue is 3 (6-3)
\end{lstlisting}
\OnehalfSpacing

Functions like \mySCName{aValue\_Function} in \myRefListing{JSTypes} do not strictly belong to the object they are declared on. Their execution context (which can be referred to using \mySCName{this} inside the function) is determined at run-time and usually refers the object through which the function is called. With the help of a function object's \mySCName{call} function, an execution context can be forced.

\SingleSpacing
\begin{lstlisting}[language=JavaScript, caption=Execution context of functions]
// obj1 declares the function
var obj1 = {
	name: "Paul",
	sayHello: function (){ return "Hello, I am " + this.name; }
}

// obj2 borrows the function (and additionally renames it)
var obj2 = {};
obj2.name = "John";
obj2.sayHello_renamed = obj1.sayHello; 

var obj3 = { name: "Ringo" };

obj1.sayHello();          // --> "Hello, I am Paul"
obj2.sayHello_Renamed();  // --> "Hello, I am John"

// forcing the execution context
obj1.sayHello.call(obj3); // --> "Hello, I am Ringo"

\end{lstlisting}
\OnehalfSpacing

\subsection{Inheritance}

The prototypical inheritance mechanism used in \myProperName{JavaScript} vastly differs from class-based inheritance known from \myProperName{C++} and \myProperName{Java}. In a class-based inheritance the class defines a design contract that created instances adhere too.\\
\myProperName{JavaScript} uses prototypes instead. A prototype is a normal \myProperName{JavaScript} object from which other objects inherit the properties. When a property is looked up on an object, the object is checked if it does have an own property with the name. If not, its prototype is checked for the property. If this doesn't have it either, the prototype's prototype is checked on so on. This is also called the prototype chain.\\
The prototype of an object is saved as an internal property ([[Prototype]]) that can not be set after object creation and can only be accessed using \linebreak\mySCName{Object.getPrototypeOf(anObject)}.\\
If a property with the same name is set on an object that has a prototype, then it shadows the prototype's property.

\SingleSpacing
\begin{lstlisting}[language=JavaScript, caption=Prototypes]
var aPrototype = { aProp: "Prop of prototype" };
aPrototype.aProp; // --> "Prop of prototype"

// creating an object with aPrototype as the prototype
var anInstance = Object.create(aPrototype);
Object.getPrototypeOf(anInstance) == aPrototype // --> true

anInstance.aProp; // --> "Prop of prototype"

// shadowing
anInstance.aProp = "Prop of instance";
anInstance.aProp; // --> "Prop of instance"
\end{lstlisting}
\OnehalfSpacing

\subsubsection{Classes}

Classes can be mimicked in \myProperName{JavaScript} with the help of constructor functions. Whenever a function is called with the \mySCName{new} operator, a new object (instance) is created. This object's internal [[Prototype]] property is set to the \mySCName{prototype} property of the function. These two are not to be confused. The \mySCName{this} keyword will refer to the created instance and can be used to define properties on the instance, for example \mySCName{name} in \myRefListing{JSClasses}. The \mySCName{sayHello} function will be shared by all instances, as it is defined on the prototype, whereas \mySCName{name} only exists on the instances themselves.

\SingleSpacing
\begin{lstlisting}[language=JavaScript, caption=Classes in \myProperName{JavaScript}, label=JSClasses]
function Person(name)
{
	this.name = name;
}

Person.prototype.sayHello = function()
{
	return "Hi, my name is (what?), my name is " + this.name;
}

var anInstance = new Person("Slim Shady");
anInstance.sayHello();
// --> "Hi, my name is (what?), my name is Slim Shady"
\end{lstlisting}
\OnehalfSpacing

For creating inheritance chains, the internal [[Prototype]] property of a function's \linebreak\mySCName{prototype} object must reference another function's \mySCName{prototype} object. If a base constructor function initializes members, then it must be called explicitly from the subclass constructor function with the help of the function object's \mySCName{call} function.

\SingleSpacing
\begin{lstlisting}[language=JavaScript, caption=Prototype-based inheritance in \myProperName{JavaScript}, label=JSClasses]
function Base()
{
	this.baseMember = 3;
}

Base.prototype.doSomething = function(){ return this.baseMember * 3;}

function SubClass()
{
	// call Base constructor with call (and without new)
	Base.call(this);
	this.subclassMember = "Another member";
}

// assigning a new prototype object with [[Prototype]] Base
SubClass.prototype = Object.create(Base);
// resetting the constructor property
SubClass.prototype.constructor = SubClass;

var instance = new SubClass();
instance.doSomething(); // --> 9 (3 * 3)

\end{lstlisting}
\OnehalfSpacing

\todo{proto chain illustration}

\subsubsection{Mixins}

Multiple inheritance is not supported in \myProperName{JavaScript}, but can be mimicked to a certain degree by borrowing functions. This concept is also called \textbf{mixin}.

\SingleSpacing
\begin{lstlisting}[language=JavaScript, caption=Mixins in \myProperName{JavaScript}, label=JSMixins]
function BaseA(){}
BaseA.prototype.funcA = function(){}

function BaseB(){}
BaseB.prototype.funcB = function(){}

function SubClass()
{
	BaseA.call(this);
	BaseB.call(this);
}

// borrowing functions / mixing in
SubClass.prototype.funcA = BaseA.prototype.funcA;
SubClass.prototype.funcB = BaseB.prototype.funcB;

// creating an instance
var instance = new SubClass();
instance.funcA();
instance.funcB();
\end{lstlisting}
\OnehalfSpacing

\section{\myProperName{Spidermonkey}}

Spidermonkey comes in form of a shared library exposing a \myProperName{C} API that can be used to interact with the \myProperName{JavaScript} interpreter. This API is also known as \myProperName{JSAPI}. 

A \mySCName{JSRuntime} is an instance of the virtual machine. A \mySCName{JSContext} is a is a single world within a \mySCName{JSRuntime}, in which all objects and values live. One \mySCName{JSRuntime} can have multiple \mySCName{JSContext}s. Each context has a global object. Objects are exposed as \mySCName{JSObject}.

A \myProperName{JavaScript} value is expressed as a \mySCName{jsval}. \mySCName{jsval} contains information about the type of the value (number, boolean, etc.). For strings and objects, the \mySCName{jsval} stores the pointer to the \mySCName{JSString} or \mySCName{JSObject}. For all other types, the data is saved in the \mySCName{jsval} itself. \mySCName{jsval}s occur at several places in \myProperName{JSAPI}, for example when defining properties, for parameter values or return values.\\
\myProperName{Spidermonkey} provides utility functions for retrieving the type of a \mySCName{jsval} \linebreak(e.g. \mySCName{JSVAL\_IS\_OBJECT}) and its content (e.g. \mySCName{JSVAL\_TO\_OBJECT}) or for creating \mySCName{jsval}s from the according \myProperName{C++} types (e.g. \mySCName{OBJECT\_TO\_JSVAL}).

A \mySCName{JSObject} has a \mySCName{void} pointer that \myProperName{JSAPI} users can use for storing any kind of private data. 


\section{Items}

Overview \myProperName{C++} and \myProperName{JavaScript}

\subsection{Boolean values}

\subsection{Number values}

\subsection{Strings}

\subsection{Functions}

\subsubsection{Overloaded functions}

\subsubsection{Default arguments}

\subsection{Structs and classes}

garbage collection, private data and jsclass jsvals

\subsubsection{Inheritance}

multiple, class-based vs prototype-based

\subsubsection{Const}

\subsubsection{Static members}

\subsection{Namespaces}

\subsection{Global variables}

\subsection{Unions}

\subsection{Function pointers and callbacks}

\SingleSpacing
\begin{lstlisting}[language=C++, caption=\myProperName{C++} function that takes a function pointer]
void someFunc(float (*funcPointer)(int));
\end{lstlisting}
\OnehalfSpacing

\myProperName{C++} functions that take function pointers, often referred to as callbacks, are a problematic topic.\\
Two cases need to be distinguished:
\begin{enumerate}
\item Passing \myProperName{C++} functions as callbacks
\item Passing script functions as callbacks
\end{enumerate}

\subsubsection{Passing C++ functions as callbacks}

The first case is unproblematic. A pointer to a \myProperName{C++} function can be wrapped inside a simple \mySCName{JSObject}, by casting the function pointer (function address) to a \mySCName{void} pointer and saving it as the private member of the \mySCName{JSObject}. The glue code for the function that takes the function pointer will retrieve the function pointer from the private data of the appropriate argument by casting it back to the appropriate function type.

\subsubsection{Passing script functions as callbacks}

When passing a \myProperName{JavaScript} function as a callback, a wrapper function that has the same function type as the pointer has to be created. This is the \myProperName{C++} function that will then be passed to the function that takes the function pointer. Inside, the \myProperName{C++} arguments must be converted to types that \myProperName{JavaScript} understands. The \myProperName{JavaScript} callback-function has to be called and its return value converted to the \myProperName{C++} type that the function type expects. Thus, such a wrapper function works the exact other way around than wrapper functions presented earlier in this chapter:

\SingleSpacing
\begin{lstlisting}[language=C++, caption=Wrapper code for handling methods that take function pointers \#1]
float wrapper_callback(int param)
{
	// 1. convert param to JavaScript Number
	// 2. call the JavaScript function
	// 3. convert the return value from a JS Number to float
}

JSBool wrapper_someFunc(JSContext *cx, uintN argc, jsval *vp)
{
	// call someFunc
	someFunc(&wrapper_callback);
}
\end{lstlisting}
\OnehalfSpacing

This approach comes with one big problem, when calling \mySCName{someFunc} multiple times with different \myProperName{JavaScript} callbacks: As \mySCName{wrapper\_callback} is a simple \myProperName{C++} function created at compile-time, it has no notion of state and as such no information about which \myProperName{JavaScript} function to use as the callback.

One way around this, is to save the \myProperName{JavaScript} callback function passed to \linebreak\mySCName{wrapper\_someFunc} in a global variable, so it can be retrieved within \mySCName{wrapper\_callback}. This approach only works, when the callback is called during the execution of \mySCName{someFunc}. If it is called at a random point in program execution, \mySCName{wrapper\_callback} would use the last \myProperName{JavaScript} function that was saved in the global variable, which may not be the one the user intended to be called. Thus this work-around is highly suboptimal.

\textbf{Passing state information}

Without run-time function creation (more about that later), a function callback of the given form can not be resolved in any other way then the previously presented. To support calling \mySCName{someFunc} multiple times with different callbacks, state information needs to be provided. This is possible in two ways, both of which need modification of the original function and/or parameter declaration.

State information can be provided by extending \mySCName{someFunc} and the function type to take an additional \mySCName{void} pointer for user data. The script function is then passed as user data to \mySCName{someFunc}, which will use/save it along with the function pointer. The user data will be passed as the \mySCName{void} pointer to the \myProperName{C} callback function and as such can be used in \mySCName{wrapper\_callback} to retrieve the script function to call.

\SingleSpacing
\begin{lstlisting}[language=C++, caption=Wrapper code for handling methods that take function pointers \#2]
void someFunc(float (*funcPointer)(int, void*), void* data);

float wrapper_callback(int param, void* data)
{
	// 1. convert param to JavaScript Number
	// 2. retrieve JSFunction to call from data
	// 3. call the JavaScript function
	// 4. convert the return value from a JS Number to float
}

JSBool wrapper_someFunc(JSContext *cx, uintN argc, jsval *vp)
{
	// retrieve JSFunction* from vp
	jsval* args = JS_ARGV(cx, vp);
	void* funcPtr =  jsval_to_jsfunction(cx, args[0]);
	
	// call someFunc
	someFunc(&wrapper_callback, funcPtr);
}
\end{lstlisting}
\OnehalfSpacing

Another way to modify the original \mySCName{someFunc} is to let it take functors that contain state information instead of a function pointer. Instead of passing the \mySCName{JSFunction} as an additional \mySCName{void} pointer, it would be saved in the functor object.

Either way, passing state information demands modification of the original source code. This may prove to be impossible in cases where 3rd party libraries are used.

\textbf{Run-time function creation and closures}

\myProperName{C++11} allows the creation of functions at run-time with the help of lambda functions\todo{ref}. Lamdba functions can capture state and thus seem to be capable of solving the given problem. This concept is also known as a closure and, as a side note, is very common in \myProperName{JavaScript}.
\\Unfortunately only state-less lamdba functions can be passed as function pointers. Lambda functions that capture state are only syntactic sugar for functors and as such can not be used to solve the given problem.

The only other way of creating functions at run-time is to create the machine-code itself at run-time. This, of course, requires assembly and is thus outside of portable \myProperName{C++}. With this method, the memory address of the \myProperName{JavaScript} function to call could simply be stored in the machine-code itself.

\textbf{libffi}\todo{ref} contains functionality for creating these types of closures at run-time. It is used by \myProperName{Mozilla}'s \myProperName{js-ctypes} for allowing \myProperName{JavaScript} functions to be passed as \myProperName{C} function pointers. \myProperName{Python}'s \myProperName{ctypes} uses it for the same reason. As the creation of closures is a highly machine-dependent operation it is indispensable to use a 3rd party like \myProperName{libffi}.

The creation of machine-code at run-time is the only way for completely supporting script functions as callbacks. It has the drawback of an additional library dependency and is also not supported on all platforms.

\subsection{Templates}

\subsection{Modules}

\subsection{Accessor Properties}

fields treated as references







\todo{TODO-List}
 - welche Sprachkonstrukte von C++ kann man auf andere Sprachen abbilden? 
 -- Klassen, Interfaces, virtuelle Methoden, Polymorphie, statische Methoden, Templates
 - Wie sieht es aus mit Sprachkonstrukten, die in C++ nicht existieren bzw. kompliziert sind?
 -- Properties? Umwandlung von Gettern und settern? Funktoren, Funktionspointer
 -- Übergang von klassen-basierter Vererbung (C++) zu Prototyp-basierter Vererbung (JS)
 - abstrahierung