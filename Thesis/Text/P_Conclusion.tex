\part{\hspace{5pt}Conclusion}
\chapter{Summary}
\label{chap:Summary}



This thesis presented the development of a highly extensible language binding application that uses the \myProperName{Clang} compiler as a static analysis tool for inspecting \myProperName{C++} files.

Extensive research was undertaken in the broad fields of language binding and static analysis. The theoretical part of this thesis aims to offer a concise overview of this research.\\
\myRefChapter{chap:LanguageBindingCPPJS} shows a practical example of how two programming languages are bridged, while presenting the research that was gathered about the \myProperName{SpiderMonkey} \myProperName{JavaScript} virtual machine and API. This chapter shows how elements of \myProperName{C++} can be exposed to \myProperName{JavaScript}, what potential problems can arise and how to work around them. A lot of effort was put into developing solutions for memory ownership related problems.

The \myProperName{CPPAnalyzer} library was developed to analyse \myProperName{C++} files. It uses the \myProperName{Clang} compiler to create a simplified abstract syntax tree that can be filtered with user-given regular expressions, while leaving the resulting tree consistent. The library is able to output its result in the form of stringified \myProperName{JSON}.

\myProperName{Clang}'s \myProperName{libclang} API was extended to expose more \myProperName{Clang} functionality, especially concerning \myProperName{C++} templates. This task required familiarization with the \myProperName{Clang} code base and its internal workings. The changes to \myProperName{libclang} can be found in a personal, though publicly available, \myProperName{git} repository and will hopefully make their way back into the official \myProperName{Clang} source code repository.

The \myProperName{Bound} GUI application allows its users to inspect a \myProperName{C++} AST and create a customized export tree for glue code generation. It currently allows the generation of glue code for binding \myProperName{C++} and \myProperName{Mozilla SpiderMonkey}.\\
Because of its dynamic and extensible nature, \myProperName{Mozilla XULRunner} was chosen as the application framework.\\
The architecture was designed with customizability and extendability in mind. With the help of language plugins and code generators for single \myProperName{C++} entities it is possible to generate glue code for multiple target languages at the same time. Extensive effort was put into the handling of types (type maps/type libraries/restrictions), error diagnosis before code generation and include-file resolution. The use of templates separates the basic wrapping logic from the actual glue code and allows end-user customization of the output.\\
The meta data system assists in loading and saving processes and allows inspection and manipulation of arbitrary \myProperName{JavaScript} objects in the GUI.\\
Combining these architectural decisions and features, other developers will easily be able to add support for other target languages by extending the basic language binding plugins and code generators. Different uses of the imported \myProperName{C++} AST, e.g. for creating type libraries or unit test skeletons, are also possible.

Being a GUI application, \myProperName{Bound} is a more user-friendly alternative to the \myProperName{SWIG} command-line tool. Although \myProperName{Bound} and \myProperName{SWIG} are very similar in that they both were developed to generate glue code, \myProperName{Bound} is taking a progressive step in how it accomplishes this task. It uses static analysis with the help of the \myProperName{Clang} compiler for compliance with future \myProperName{C++} standards. Building upon the \myProperName{JavaScript} based extendable \myProperName{XULRunner} as an application framework allows easier development and contribution.

In conclusion, the majority of my personal aims for this thesis have been accomplished. An extendable and customizable glue code generation tool has been developed, which is able to create functioning glue code for \myProperName{Mozilla SpiderMonkey}. 

Due to time constraints, the original plan of supporting a second target language like \myProperName{Python} or \myProperName{Lua} had to be dropped.\\ The wrapping for \myProperName{SpiderMonkey} was also not completed. However, in its current state, it is possible to wrap namespaces, classes, inheritance chains and (member) functions.\\
With the basic architecture complete, full support of
further elements (data fields, constructors, templates, etc.) could be possible to incorporate with relative ease in the future.

\chapter{Future Development}
\label{chap:FutureDevelopment}

\myProperName{Bound} is currently in alpha status. There are still ways of using the application in a manner that leads to undefined behaviour. Creating a stable application, fixing these shortcomings and enhancing the usability of the GUI should be the first steps taken in further development.

Concerning language plugins, the \myProperName{SpiderMonkey} plugin should be completed, before new target languages are added.

The application should also be tested to confirm whether auto-generated wrapper code for real world libraries behaves correctly in all situations.

As forementioned, the changes in my personal \myProperName{Clang} repository should be merged into the official source code base. Otherwise \myProperName{Bound} would rely on an unofficial \myProperName{Clang} build.

\hspace{20pt} Custom \myProperName{Clang}: \url{https://github.com/FunkMonkey/libClang}

In \myRefChapter{chap:DesignGoals} it was stated that the project is being developed as open-source software.\\
The source code is publicly available on \myProperName{github}:

\hspace{20pt} \myProperName{Bound} source code: \url{https://github.com/FunkMonkey/Bound}

Developing an open-source project is more laborious than simply publishing the source code. Documentation has to be written and the source code has to be made compilable on platforms other than \myProperName{Microsoft Windows}.\\
Developers, who wish to contribute, may seek assistance, when they are having problems with the source code or with understanding the architectural design of the application.\\
Also, a software license must be chosen.

Writing language binding glue code by hand is an arduous and time-consuming task, which is highly prone to human error. The developed application will assist the \linebreak programmer immensely in the creation of such code and tailor the code's behaviour according to the programmer's needs.