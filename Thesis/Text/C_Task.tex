\chapter{Task}
\label{chap:Task}

After introducing the basic concepts behind language binding, this chapter will clarify what kind of application will be developed, as well as present the goals and non-goals concerning the development.

The application to be developed will be released as open-source.

The purpose of the application is to create high-performance glue code for binding different scripting languages to C++. The export will happen as a static foreign interface that needs to be included and compiled in the user's program.

The application will be developed in a way that is easily extendable to support a multitude of different scripting languages, starting with support for Mozilla Spidermonkey (ECMAScript/JavaScript). Furthermore it should also be extendable for other uses of the analysed C++ syntax tree, e.g. for documentation or analysing purposes, creating type libraries or generating skeleton code for unit tests.

The application needs to be able to create language bindings without the need to modify the original code.

Furthermore it should give the user the potential to customize and configure the process of language binding to suit the produced code to his or her individual needs.

The necessary information about the symbols and types declared in the provided C++ header files will be retrieved and reprocessed using a self-written static analysis tool that uses the Clang C/C++ compiler internally.

Other long term goals, outside of the time interval of this thesis, include the support for other scripting languages, a language module for exporting SWIG interfaces, enhanced usability and the support of input languages different from C/C++.