\chapter{GUI application}
\label{chap:GUIApplication}

The GUI application is based on \myProperNameImp{XULRunner}\footnote{\url{https://developer.mozilla.org/en/XULRunner}}, an application framework developed by \myProperName{Mozilla} and as such also known as the Mozilla Framework. \myProperName{XULRunner} also serves as the foundation of popular applications like \myProperName{Mozilla Firefox}\footnote{\url{http://www.mozilla.org/en-US/firefox/}} and \myProperName{Thunderbird}\footnote{\url{http://www.mozilla.org/projects/thunderbird/}}. 

\myProperName{XULRunner} applications are mainly created using \myProperNameImp{XUL}\footnote{\url{https://developer.mozilla.org/en/XUL}} (XML User Interface Language - a markup-language similar to \myProperNameImp{HTML}\footnote{\url{http://www.w3.org/html/}}), \myProperNameImp{CSS}\footnote{\url{http://www.w3.org/Style/CSS/Overview.en.html}}(Cascading Style Sheets) and \myProperNameImp{JavaScript} (\myProperNameImp{ECMAScript}\footnote{\url{http://www.ecmascript.org/}}). \myProperName{XUL} can seamlessly be replaced with \myProperName{HTML} for markup, thus \myProperName{XULRunner} applications can be build using the same technologies as web pages. Unlike web pages though, \myProperName{XULRunner} is a full featured desktop application framework and as such provides functionality to access the operating system, for example for file manipulation. Most of this functionality can be accessed from \myProperName{JavaScript} using \myProperNameImp{XPCOM} (see \myRefSection{sec:ComponentModels}). \myProperName{XULRunner} can be extended with native code through \myProperName{XPCOM} binary components or access of shared libraries using \myProperNameImp{js-ctypes} (see \myRefSection{sec:DynamicFFI}).

The decision to create the GUI application using \myProperName{XULRunner} is mostly based on its easy extendability and customizability - both stated as design goals in \myRefChapter{chap:DesignGoals}. Similar to the \myProperName{Firefox} browser, users will be able to write extensions to provide support for new features, for example glue code generators for scripting languages not supported by the main application.\\
Being based on web technologies, especially \myProperName{JavaScript}, development with  \myProperName{XULRunner} is very fast.
\\Personal preference and experience with the framework have also been major factors.

\section{GUI overview}

\section{Basic architecture}

javascript modules, Bound-module, metadata

\section{\myProperName{C++} AST}
\subsection{Calling CPPAnalyzer}


\subsection{Export AST}

code generators, templates, 
    TypeSystem
        Shadowing of templates
    Templates
        templating using jSmart
        templates in JSON for additional data
        problem with escaping newline and tab
        includes
        custom functions (die auch auf alles zugreifen können, z.b. log) -> fungieren genauso als return-values!

type printer
\subsection{The Spidermonkey code generator}

type libraries

\subsection{Project}

saving and loading, metadata, etc.

\section{GUI widgets}

\subsection{ObjectExplorer}

\subsection{DOMTree}

\section{Known Problems}

from which file was a symbol included: need include file hierarchy


