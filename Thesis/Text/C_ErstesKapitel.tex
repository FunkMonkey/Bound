\chapter{Erstes Kapitel}
\label{erstes_kapitel} % label um in der PDF auf dieses Kapitel verlinken zu können

Hier ein Beispiel für ein Link zu einem anderen Teil des Dokuments. Bla bla nähere Informationen in \myRefChapter{erstes_kapitel}. Nicht wundern: Das Latex-Dokument muss mehrmals kompiliert werden, damit manche Sachen, wie diese Links oder das Inhaltsverzeichnis korrekt angezeigt werden. Das ganze geht auch mit Abschnitten: \myRefSection{ersterabschnitt}

Hier ist eine todo\todo{Was ist zu tun?}-Notiz. Latex hat übrigens Probleme mit Unterstrichen. Muss man halt mit Backslash escapen. Für Anführungszeichen nimmt man ''zwei Apostrophe''. Standard-Anführungszeichen sorgen irgendwie automatisch für Umlaute.

\section{Erster Abschnitt}
\label{ersterabschnitt}

\begin{lstlisting}[language=C, caption=Code Beispiel]
#include <stdio.h>

// Hier ein Code-Beispiel
int init(const int& x)
{
	return x + 3;
}
\end{lstlisting}

\subsection{Subabschnitt}

\begin{figure}[h] % h = here
	\centering
		\includegraphics[scale=0.6]{Images/alpha_bug.png}
	\caption{Bezeichnung des Bildes}
\end{figure}

\newpage
\subsubsection{Subsubabschnitt}

Wenn ein Bild mal breiter sein soll als der Text. Passiert nicht oft, aber kann ja mal vorkommen.

\begin{figure}[h] % h = here, makebox for wider images
	\makebox[\textwidth]{%
		\includegraphics[width=1.2\linewidth]{Images/alpha_bug.png}}
	\caption{Sehr breites Bild}
\end{figure}