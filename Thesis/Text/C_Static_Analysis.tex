\chapter{Static Analysis}

In ''Secure Programming with Static Analysis'' \textbf{static analysis} is defined as followed:

\begin{quotation}
The term static analysis refers to any process for assessing code without
executing it. Static analysis is powerful because it allows for the quick consideration of many possibilities. A static analysis tool can explore a large number of ''what if'' scenarios without having to go through all the computations
necessary to execute the code for all the scenarios.\footnote{\citep[3]{SecureProgramming}}
\end{quotation}

Another definition that highlights the use of prediction during the process of analysis can be found in ''Principles of program analysis'':

\begin{quotation}
Program analysis offers static compile-time techniques for predicting safe and computable approximations to the set of values or behaviours arising dynamically at run-time when executing a program on a computer.\footnote{\citep[1]{ProgramAnalysis}}
\end{quotation}

Thus, tools performing static analysis evaluate software in the abstract, without running the software or considering a specific input.\footnote{\citep[1]{UsingSAToFindBugs}}

Complementary to static program analysis, there is also \textbf{dynamic program analysis}, which analyses software at run-time. Unit testing comes to mind as a prominent example of dynamic program analysis.

\section{Area of application}

The general purpose of static analysis is to extract useful information from source code or compiled code (like byte code or binary code).

The type of information retrieved depends on the context and purpose of the static analysis and can be arranged into different categories:

\begin{itemize}\addtolength{\itemsep}{-0.5\baselineskip}
\item Program correctness
\item Software security
\item Style checking
\item Program understanding
\item Optimization
\end{itemize}

\subsection{Program correctness}

Static analysis tools that are concerned with program correctness try to detect defects at compile-time so that those do not manifest in run-time errors, such as crashes, data corruption and program malfunctioning. Thus these tools, which are mostly concerned with \textbf{software quality}, are often referred to as \textbf{bug finders}.

Bug finders point out places in the source code, where the program will behave in a way that the programmer did not intend. Most tools are easy to use because they come pre-stocked with a set of ''bug idioms'' (rules) that describe patterns in code that often indicate bugs.\footnote{\citep[32]{SecureProgramming}} They help to find these often hard-to-spot defects early in the software development life-cycle, reducing the cost, time, and risk of software errors.\footnote{\citep{CovertySA}}

A prominent example of preventing run-time errors at compile-time is type checking\todo{definition}. In this sense, compilers themselves can be seen as static analysis tools and - as we will see later - both have much in common.

''The lint program for C programs is generally considered the first widely used static analysis tool for defect detection.''\footnote{\citep[1]{UsingSAToFindBugs}}

\subsubsection{Generic and context-specific defects}

Defects can be categorized in generic and context-specific defects.

\textbf{Generic defects} are problems that can occur in almost any program written in the given programming language, such as buffer overflows and memory leaks. \textbf{Context-specific defects} on the other hand require specific knowledge about the semantics of the given program.\footnote{\citep[14]{SecureProgramming}} Applications that deal with atomic operations (like a program that handles money transfer) belong to the latter category.

The following sections will present different kinds of bugs, most of which can be found by analysing the control and data flow of an application.

\subsubsection{Overflow and range analysis problems}

A buffer overflow occurs when a program writes data outside the bounds of allocated memory\footnote{\citep[175]{SecureProgramming}}, for example when writing data into a buffer that is too small for the data to be written, leading to the memory after the buffer being overwritten and thus corrupted. Static analysis tools can detect such problems and advice the programmer to do a bounds-check.\todo{more internals}

An integer overflow occurs when an integral value is increased or decreased beyond its capacity\footnote{\citep[235]{SecureProgramming}}. The following example demonstrates how this can lead to an infinite loop:

\begin{lstlisting}[language=C++, caption=Integer ''underflow'' in C++]
for(unsigned int i = 200; i >= 0; --i)
{
	// ...
}
\end{lstlisting}

A static analysis tool will detect that the expression \textbf{i \textgreater= 0} will always be true for \textbf{unsigned} integer types like \textbf{i}, as i will be set to the highest possible unsigned integer when being 0 and decreased by 1.

\subsubsection{Resource leaks}

Resource leaks can be detected by tracking request (for example allocation of memory) and release of resources.

If a program gives up all references to a resources while it is has not been released, the resource is leaked. On the other hand, using a resource that has not been requested properly or releasing an already released resource may lead to unexpected results or even crashes.

The most prominent example of a resource leak is a memory leak. But there are other important resources like files and databases that may need to be locked before usage and released afterwards.

\subsubsection{Threads and concurrency}

Static analysis tools can also be used to find problems with threads such as data races and deadlocks.

Similar to the detection of resource leaks, the data and call flow of the application can be analysed to track the locking and unlocking of semaphores.

\subsubsection{Other problems and warnings}

There are many more possible sources for bugs that can be detected using static analysis tools, such as division by zero, dereferencing of null-pointers or the returning of references or pointers to local function variables.

Bug finders are not only concerned with finding malicious code that will lead to run-time errors, but may also warn about code that is redundant, like comparisons that will always have the same result. Although such a comparison won't cause a failure or exception, its existence suggests that it might have resulted from a coding error, leading to incorrect program behaviour.\footnote{\citep[1]{UsingSAToFindBugs}}

Similar warnings could be issued for functions whose\todo{whose?} return values (for example error codes) have not been checked or for exception catch-clauses that are too broad, so that they catch important exceptions like OutOfMemoryError without handling them.

\subsection{Software security}

\begin{quotation}
Static analysis is particularly well suited to security because many security problems occur in corner cases and hard-to-reach states that can be difficult to exercise by actually running the code.\footnote{\citep[4]{SecureProgramming}}
\end{quotation}

Concerning software security, static analysis tools are used to find coding errors before they can be exploited. Buffer overflows and format string vulnerabilities come to mind, possibly resulting in SQL injection, cross-site scripting or unwarranted acquisition of administrator privileges. 

Static analysis may also track input data flow and validation to determine
all the implicit ways a program might be putting unwarranted faith in some aspect of its input.\footnote{\citep[172]{SecureProgramming}}

\subsection{Style checking}

Style checkers enforce rules related to whitespace, position of scope-brackets, naming conventions, deprecated functions, commenting, program structure, and similar non-semantic issues.\footnote{\citep[25]{SecureProgramming}}

\todo{picture}

\subsection{Program understanding}

Program understanding tools help users make sense of a large codebase. \footnote{\citep[27]{SecureProgramming}} Most modern IDE's provide functionality for helping the user navigating the codebase, for example jumping to a functions definition, finding all uses of a method. Some even provide functionality for refactoring symbols, like renaming classes. All these operations need knowledge of the symbols existing in the codebase.

Other tools visualize the source code, for example by creating UML diagrams from source, rendering images of class hierarchies or function callgraphs that can help the user understand the control flow of the application and the relationship of the existing symbols.

\todo{picture}

Prominent tools like Doxygen\footnote{\url{http://www.stack.nl/~dimitri/doxygen/}} automatically create documentation from source code and code comments.

\subsection{Optimization}

Compilers perform static analysis to do\todo{word} code optimization:

\begin{quotation}
A main application is to allow compilers to generate code avoiding redundant computations, e.g. by reusing available results or by moving loop invariant computations out of loops, or avoiding superfluous computations, e.g. of results known to be not needed or of results known already at compile-time.\footnote{\citep[1]{ProgramAnalysis}}
\end{quotation}

Apart from code optimization, compile-time optimization, for example by finding unnecessary Include-files in C++ applications is another area that static analysis is used for. \textbf{include-what-you-use}\footnote{\url{http://code.google.com/p/include-what-you-use/}} is a tool to perform this task. It internally uses the Clang compiler.

\section{Static analysis as part of a programmers workflow}

Being major factors of software quality, companies put a lot of effort into improving and verifying \textbf{program correctness} and \textbf{software security}. Different techniques have been developed, each concerned with achieving this main goal, but taking varying routes and targeting different stages of the development.

\textbf{Software engineering} and planning begin long before any program code is written, specifying program behaviour and design, and finding structural problems and security concerns before implementation. \textbf{Pair programming} aims at reducing coding errors and making design choices at the time of writing program code by using human experience and communication. \textbf{Code review} is similar, but happens at a later stage. \textbf{Dynamic testing}, including unit tests, has been invented to verify correct run-time program behaviour of written code programmatically, with the help of more program code that tests code and compares the results to the expected output.

All these techniques follow a defensive mindset about writing applications, knowing that every programmer, no matter how experienced, will make mistakes that lead to undefined program behaviour. As such these techniques aim at reducing the rate of errors that go unnoticed.

\textbf{Static analysis} is another technique in the programmers toolbox serving this goal, during and after code creation. Compared to dynamic testing and code review, it reduces the human factor, programatically applying ''checks thoroughly and consistently, without any of the bias that a programmer might have about which pieces of code are 'interesting' (...) or which pieces of code are easy to exercise through dynamic testing.''\footnote{\citep[22]{SecureProgramming}} Static analysis tools accumulate the knowledge of experienced programmers and as Brian Chess and Jacob West point out, the feedback given by these tools acts as a means of knowledge transfer by showing the programmer types of errors he was not aware of.\footnote{\citep[22]{SecureProgramming}} As such, a "static analysis tool can make the code review process faster and more fruitful by hypothesizing a set of potential problems for consideration during a code review."\footnote{\citep[13]{SecureProgramming}}

As with dynamic testing and code review, static analysis can not verify the absence of errors, but is a technique to reduce the amount of bugs before release.

\subsection{Drawbacks}

As every technique, static analysis has drawbacks. 

Depending on how well a tool works and how good it is balanced, static analysis tools may produce a lot of \textbf{false positives}, meaning pointing out an error where none exists. Too much ''noise'' is problematic, as it costs the programmer valuable time to check if a result is really a bug or not. Also, real problems may be overseen, when false positives dominate the tools output.

\textbf{False negatives}, on the other hand, are real bugs that a tool did not discover. It has already been said, that complete absence of errors can not be proven with any tool - as such using static analysis tools may give the programmer a false sense of security.\footnote{\citep[23]{SecureProgramming}}


\section{Static analysis tool internals}

To retrieve useful information from the code, static analysis tools need to build an internal representation (model) of the program to be analysed. The structure of that model heavily depends on the purpose of the tool and has to make good trade-offs between precision, depth, and scalability.\footnote{\citep[45]{SecureProgramming}} After such a model has been created, the tool can use the information for further processing. It could, for example, perform rule-based checks on the model to find coding errors.

\subsection{The model}

When creating the internal program model, static analysis tools generally borrow a lot from the compiler world\footnote{\citep[72]{SecureProgramming}}. They especially follow the first two phases of a compiler: lexical analysis and syntax analysis.

\begin{quotation}
\textbf{Lexical analysis} (...) is the initial part of reading and analysing the program text: The text is read and divided into \textit{tokens}, each of which corresponds to a symbol in the programming language, e.g., a variable name, keyword or number.\footnote{\citep[2]{CompilerBasics}}
\end{quotation}

Compared to compilers, static analysis tools may filter out less during lexical analysis. They may preserve comments, in case the tool needs information from comments for the model (Doxygen is a prominent example for this case).
Static analysis tools that search for coding errors usually also associate the tokens with their source code positions for later error reporting.\footnote{\citep[72]{SecureProgramming}} 

\begin{quotation}
\textbf{Syntax analysis} (...) takes the list of tokens produced by the lexical analysis and arranges these in a tree-structure (called the \textit{syntax tree}) that reflects the structure of the program. This phase is often called \textit{parsing}.\footnote{\citep[2]{CompilerBasics}}
\end{quotation}

As the created syntax tree contains a lot of irrelevant data that is not needed for later analysis (parentheses, etc.), the syntax tree is usually transformed into an \textbf{abstract syntax tree (AST)}, where such information is either removed or combined from multiple tree nodes into single ones.\footnote{\citep[99]{CompilerBasics}}

Some static analysis tools may go their own way from this point on and create their model based on the abstract syntax tree.\\Other tools, like bug finders, as well as compilers for statically typed languages will perform another phase common in the compiler world: \textbf{semantic analysis}. During that phase, a symbol table will be build, which associates each identifier found in the program with its type and a pointer to its declaration or definition.\footnote{\citep[76]{SecureProgramming}} Using the created table, the tool or compiler is able to perform type-checking to check the correctness of the program.\footnote{\citep[76]{SecureProgramming}}

Bug finders additionally need to populate their model with information about the effects of library or system calls invoked by the program being analysed.\footnote{\citep[37]{SecureProgramming}}

\subsection{Applying rules}

After creating a model, static analysis tools that check for correctness, will perform checks on the model using a set of rules that are delivered with the tool. These rules may search for common ''bug idioms'' like memory leaks, buffer overflows or may define the characteristics of a special security related problem.

Some tools allow the user to extend the set of rules to search for bugs that may be specific to the users codebase or that are not found by the rules delivered with the tool.\footnote{\citep[97]{SecureProgramming}}

For altering the output of the static analysis tool, some bug finders allow the annotation of source code passages.\footnote{\citep[99]{SecureProgramming}} An annotation can be helpful to suppress a false positive in the output.
\\For languages that do not provide special syntax for annotations, as Java and C\# do, annotations usually
take the form of specially formatted comments.\footnote{\citep[99]{SecureProgramming}}

The approaches that are used for checking rules differ from rule to rule. Some checks track control flow or data flow throughout the program. Others use alternative relationships that can be drawn from the model. How static analysis tools perform these checks is out of scope of this thesis. For further knowledge the reader is advised to gather information about \textit{Control Flow Analysis}, \textit{Data Flow Analysis}, \textit{Constraint Based Analysis}, \textit{Taint Propagation}, \textit{Abstract Interpretation}, \textit{Interprocdural Analysis }as well as \textit{Type and Effect Systems}. 