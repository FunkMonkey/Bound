\chapter{Introduction}

This master's thesis deals with the design and development of an open-source GUI application called \myProperNameImp{Bound}. This application creates glue code for connecting different programming languages like \myProperName{C++} and \myProperName{JavaScript} within in a single program - a process known as \textbf{language binding}.\\
It uses \textbf{static analysis} to retrieve the necessary information regarding symbols from \myProperName{C++} header files with the help of the \myProperName{Clang} \myProperName{C/C++} compiler.

The application has been developed in order to support a multitude of different programming languages. In this thesis support will be added for binding C++ and \myProperName{SpiderMonkey} (\myProperName{Mozilla}'s implementation of the \myProperName{ECMAScript} standard).

This thesis will introduce the basic concepts behind language binding and static analysis. It will further explain the process from analysing source code files to exporting language binding code.

\section{Overview}

\myRefChapter{chap:LanguageBinding} defines the term language binding and presents different ways in which programming languages can be bound together within a single application.

\myRefChapter{chap:LanguageBindingCPPJS} gives insights about the different kinds of constructs the two programming languages \myProperName{C++} and \myProperName{JavaScript} provide and how these can be mapped onto each other.

In \myRefChapter{chap:StaticAnalysis} basics of static program analysis are described and areas of application shown. The chapter will also present existing static analysis tools and explain the decision to use \myProperName{Clang} for performing static analysis.

Given the information from the previous chapters, \myRefChapter{chap:DesignGoals} will formulate design goals for the application to be developed.

\myRefChapter{chap:AnalysingCPP} shows details about the implementation of a \myProperName{C++} application that analyses certain symbols and declarations from \myProperName{C} and \myProperName{C++} source code files and exports the information as \myProperName{JSON}.

The exported information will be used in a GUI application based on the \myProperName{Mozilla Framework}, as described in \myRefChapter{chap:GUIApplication}. The chapter will introduce the basic architecture of the application and explain how the symbol information is used to create language binding code for \myProperName{JavaScript}.

\myRefChapter{chap:Summary} and \myRefChapter{chap:FutureDevelopment} form the \textbf{conclusion}.

\section{Formatting}

Proper names of companies and products (e.g. \myProperName{Mozilla},  \myProperName{Firefox} or \myProperName{JavaScript}) are written in capital letters.

Source code passages are presented in special boxes:

\SingleSpacing
\begin{lstlisting}[language=JavaScript, caption=Example source code]
function aFunctionName()
{
	return "bar";
}
\end{lstlisting}
\OnehalfSpacing

For source code identifiers that are used in text, e.g. \mySCName{aFunctionName}, a monospaced font is used.

\myLBHighlight{Passages that are highlighted with a left blue border can be found in earlier chapters and indicate information about the goal of this thesis and the developed application.}




