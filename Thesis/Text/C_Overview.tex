\chapter{Introduction}

This masters thesis deals with the design and development of an open-source GUI application called \myProperNameImp{Bound} that creates glue code for connecting different programming languages like \myProperName{C++} and \myProperName{JavaScript} within in a single program - a process known as \textbf{language binding}.\\
The application uses \textbf{static analysis} to retrieve the necessary information about symbols from \myProperName{C++} header files using the \myProperName{Clang} \myProperName{C/C++} compiler.

The application is developed in a way that it can support a multitude of different programming languages, starting with the support of binding \myProperName{C++} and \myProperName{Mozilla's ECMAScript} implementation \myProperName{Spidermonkey} as presented in this thesis.

The thesis will introduce the basic concepts behind language binding and static analysis, and further explain the process from analysing source code files to exporting language binding code.

\subsection{Overview}

\myRefChapter{chap:LanguageBinding} defines the term language binding and presents different ways in which programming languages can be bound together within a single application.

\myRefChapter{chap:LanguageBindingCPPJS} gives insights about the different kinds of constructs the two programming languages \myProperName{C++} and \myProperName{JavaScript} provide and how these can be mapped onto each other.

In \myRefChapter{chap:StaticAnalysis} basics of static program analysis are described and areas of application shown. The chapter will also present existing static analysis tools and explain the decision to use \myProperName{Clang} for performing static analysis.

Given the information from the previous chapters, \myRefChapter{chap:DesignGoals} will formulate design goals for the application to be developed.

\myRefChapter{chap:AnalysingCPP} shows details about the implementation of a \myProperName{C++} application that analyses certain symbols and declarations from \myProperName{C} and \myProperName{C++} source code files and exports the information as \myProperName{JSON}.

The exported information will be used in a GUI application based on the \myProperName{Mozilla Framework}, as described in \myRefChapter{chap:GUIApplication}. The chapter will introduce the basic architecture of the application and explain how the symbol information is used to create language binding code for \myProperName{JavaScript}.

\myLBHighlight{Passages that are highlighted with a left blue border can be found in earlier chapters and indicate information about the goal of this thesis and the developed application.}


